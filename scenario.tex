\section{Using \systemname{}}

\andrew{Create some graphics here of tutorials before and after tutoroids, \emph{highlighting} the text that is created by tutoroids.}.

\subsection{Following tutorials augmented with \glspl{name}}

\user{} decides that \userpro{} wants to set up a web scraper that collects the top music albums on Amazon.com every day and save the data to a text file.
\userpro{} has never written a scraper before, but has used Python for scientific programming, so \userpro{} feels pretty confident \userpro{} can pick up this skill pretty quickly with the right guide.
\userpro{} opens up \userpos{} browser to Google, and types in `Python web scraping' as the query.
After scanning the results, \userpro{} selects the page titled ``Web Scraping 101 with Python''\footnote
{This is an actual tutorial hosted online at \url{http://www.gregreda.com/2013/03/03/web-scraping-101-with-python/}}.

Well, things don't get off to the best start.
\andrew{Probably this should begin with a programming example to show the real essence of \glspl{name}.
Then we could add a note about more qualitative, background-type stuff like starting Terminal after this discussion.}
\user{} typically uses IDLE, but this tutorial asks the user to open up the Terminal.
And it's looking like the tutorial follows some steps that uses Terminal in a way that \userpro{} should probably follow --- \userpro{}'s not sure how to do \emph{setup easy\_install} or \emph{pip install} from IDLE.
After looking around in LaunchPad for a bit, \userpro{} can't find Terminal and isn't sure what to do next.
\userpro{} clicks on the phrase \emph{open up Terminal} in the text, and sees an extra sentence get added beneath the current paragraph.
\andrew{Put a figure here.}
It reads ``If on OSX Mavericks or later, click on the LaunchPad icon in the dock, click on the icon that reads `Other' on the first page, and choose `Terminal' from the list that appears.''

\andrew{Can we emphasize developers making each others' documentation more effective?}

\subsection{Using \gls{name}-augmented tutorials as starting points}

Of course, just following the tutorial is enough.
Sure, \user{} can folow the guidelines in the tutorial and now understands how the author scraped data from the \emph{Chicago Reader's Best of 2011} list.
But \user's original task was to scrape top albums from Amazon.com.
It's time for him to start sleuthing to find out how he can adapt the author's final code to his own purposes.

\andrew{What if we can do chaining to the extent where \user{} can understand how to use CSS selectors instead of ``lxml" from start to finish, even though CSS isn't mentioned in the original tutorial?}
\andrew{Can we produce argument-mining from the web, where we look for common arguments, and potentially descriptions on the web, for providing something like all the possible parsing formats (`lxml', `css') for initializing BeautifulSoup?}

\subsection{Writing a \gls{name}}

\subsubsection{One-off authoring}

Users select right click a point in the text where they want to intervene.
The text immediately becomes editable.
While users are unable to delete any existing text, they can add any text that they want.
However, they are limited to doing so in \emph{300 characters or less}.
\andrew{This might be absurd, but it's interesting to think about how we can limit users to playing well within the original tutorial.}
\andrew{Also, perhaps these user-initiated markups should be optional notes, like the highlights and comments you can see from Kindle --- hide them away but let users know that they're there if they're facing errors.}
In the future, sites could choose to include setup scripts that enable them to be user-editable.
For now, we implement this through a guerilla-editing perspective -- anyone with the \Glspl{name} extension can edit pages in a way that can be seen by anyone else with the extension.

\subsubsection{Writing general-purpose \glspl{name}}

You have to define 4 different parts:
First, define a rule to detect the use of the skill in the code.
Second, define (optional) rules to detect description of the skill in the text.
\andrew{Probably this will be a trick we show at the end of the paper instead of up front, as a feasibility confirmation and a call out to future work.}
Third, define natural language generation rules.
Fourth, define the place in a tutorial that you would like to place the augmentation.
