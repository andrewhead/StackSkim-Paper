\section{Using \Glspl{name}}

\andrew{Create some graphics here of tutorials before and after tutoroids, \emph{highlighting} the text that is created by tutoroids.}.

\subsection{Writing a \gls{name}}

You have to define 4 different parts:
First, define a rule to detect the use of the skill in the code.
Second, define (optional) rules to detect description of the skill in the text.
Third, define natural language generation rules.
Fourth, define the place in a tutorial that you would like to place the augmentation.

\subsection{Following tutorials augmented with \glspl{name}}

\user{} decides that \userpro{} wants to set up a web scraper that collects the top music albums on Amazon.com every day and save the data to a text file.
\userpro{} has never written a scraper before, but has used Python for scientific programming, so \userpro{} feels pretty confident \userpro{} can pick up this skill pretty quickly with the right guide.
\userpro{} opens up \userpos{} browser to Google, and types in `Python web scraping' as the query.
After scanning the results, \userpro{} selects the page titled ``Web Scraping 101 with Python''\footnote
{This is an actual tutorial hosted online at \url{http://www.gregreda.com/2013/03/03/web-scraping-101-with-python/}}.


\subsection{Using \gls{name}-augmented tutorials as starting points}

Of course, just following the tutorial is enough.
Sure, \user{} can folow the guidelines in the tutorial and now understands how the author scraped data from the \emph{Chicago Reader's Best of 2011} list.
But \user's original task was to scrape top albums from Amazon.com.
It's time for him to start sleuthing to find out how he can adapt the author's final code to his own purposes.

\andrew{What if we can do chaining to the extent where \user{} can understand how to use CSS selectors instead of "lxml" from start to finish, even though CSS isn't mentioned in the original tutorial?}
\andrew{Can we produce argument-mining from the web, where we look for common arguments, and potentially descriptions on the web, for providing something like all the possible parsing formats (`lxml', `css') for initializing BeautifulSoup?}
