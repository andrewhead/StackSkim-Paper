\section{Scenario}

\user{} is looking for a middleware that will log database queries for all HTTP requests to his Django server to a local file.
\andrew{
This use case may be ridiculously specific.
And this workflow might not work for all queries, which might not have appropriate results.
}
As this is a secondary programming task for the day, he looks to Google to provide an implemented solution.
He opens up his browser and enters a search query for `Django middleware log SQL query to file'.
This directs him to the Github page for \emph{django-queryinspect}, a package that includes the middleware that he needs.
But as his server is going to handle over 100 queries per second in peak hours, he wants to take precautions prior to incoporating this middleware into his code.
The \systemname{} plugin helps him to verify that it meets some of his bare requirements.

By highlighting the code in the browser, the \systemname{} plugin starts automatically in a side panel.
In the header for the code, \user{} sees that this code depends on the \texttt{django.conf}, \texttt{django.db}, \texttt{django.db.backends.util}, and \texttt{django.core.exceptions} packages.
This tells him the packages that he must have installed in order to run the middleware.
This is a start for him to know whether the libraries relied upon are trustworthy.
In addition, he sees a number of coding schemes that he wants to inspect sequentially in order to determine that the code is safe to use:

\emph{Does the code use libraries the right way?}
Each line that invokes an external function is annotated with a ``+'' button on its right.
By clicking on this button, \user{} expands an informational dialog that provides an example of the function in use from its API.
In addition, each argument is mapped from the name he provides to the parameter name in the documentation.
By clicking on ``See more...'', he can see a brief description of the method as well as descriptions of the parameters and return values.

\emph{Where is this code error-prone?}
A line of code is drawn in red if it has the potential to throw an exception.
A line of code is drawn in yellow if it has not passed a linting test.
\user{} can edit the code in place to handle these errors, causing the red and yellow to disappear.

\emph{How fast will this code run?}
As \user{} is attempting to satisfy 100 queries per second, it's important that there be no operations that take longer than a few $ms$ to complete.
By clicking on the `Timing' tab, the coloring of the lines change.
Lines in red take a long time to compute.
Lines in green take a relatively trivial time to run.
The time in $ms$ to compute is shown on the right-hand side of each line of code.
\user{} edits a few of the statements in the code, seeing their color change from red to green as he decreases their runtime to something simpler.

\emph{What can I use besides this code?}
By clicking on the ``What are my alternatives?'' button at the bottom of the panel, \user{} is shown 3 examples that semantically match the content of the code he has seen, taken from StackOverflow and other online tutorials.
Each of these is colored according to risk of error or timing, according to the coloring he has chosen for the code he inspects.
In this way, he can compare the relative performance and risk of error between the original code he found and the alternatives.

Additional features not described in this tutorial include:
\begin{itemize}
\item Inserting additional variable dependencies if pulling incomplete code from an example.
\item Showing what else is needed to make an incomplete snippet compile.
\end{itemize}
