\section{Definitions}

Throughout this paper,  we refer to the following terms to describe the system, and its input and output:

A \emph{\Gls{name}} is a routine on a web server with language-specific rules for detecting, parsing and explaining source code written on a web page.

An \emph{explainable region} is a substring of an online code snippet that a \Gls{name} can parse and explain.  For example, the string \texttt{div.klazz} in the JavaScript code \texttt{\$(\qs{}div.klazz\qs{}).text(\qs{}foo\qs{})} is an explainable region for the CSS selector \Gls{name}.

A \emph{\gls{exp}} is the output of a \Gls{name} for an explainable region of code. This can be a natural language explanation or a domain-specific demonstration of what the code does.

%%A \emph{code snippet} is a block of code on a page encompassed in a single DOM element --- e.g., a \texttt{<code>} or \texttt{<pre>} block --- that a programmer is taking effort to understand or reuse.

The \emph{\Glspl{name} addon} is a browser extension that queries \Glspl{name} for \glspl{exp} of code and displays them on-demand for explainable regions.  We use the addon to demonstrate the advantages of context-relevant, on-demand in-situ programming help.
