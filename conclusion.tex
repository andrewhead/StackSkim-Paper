\section{Conclusions}

Programmers frequently turn to the web to solve coding problems and find example code.
\begin{changes}
\Glspl{name} produce explanations for unexplained code programmers find on the web.
In our framework, \Glspl{name} can produce effective \glspl{exp} when they leverage multiple representations, focus on dynamically generated content, build on existing documentation, and address common usage.
We show that \Glspl{name} for a collection of languages achieve around 80\% accuracy detecting code examples, and improve programmers' ability to modify online code examples without referencing external documentation.
\Glspl{name} bring context-relevant, on-demand documentation to the code examples programmers encounter in online programming documentation.
\end{changes}

\if 0
However, code snippets within  help documentation can contain unexplained concepts that require programmers to seek additional help.
We propose language-specific routines called \Glspl{name} that automatically generate \emph{on-demand, context-relevant} micro-explanations of code.
We demonstrate the technical effort required to build \Glspl{name} by making \gls{exp} for CSS selectors, regular expressions, and Unix commands.
Through a preliminary in-lab study with programmers, we show that these automatically-generated explanations improve participants' ability to modify online code examples without referencing external documentation.
We believe that the \Gls{name} ecosystem will make onine programming examples more understandable and resuable.
\fi
