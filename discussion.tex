\section{Discussion}

StackOverflow and Google are already usable, functional interfaces for finding programming solutions.
however, \systemname{}'s unique strength arises when there are multiple ways to solve a problem and many online resources describing how to do it.
A breadth-first consideration could be helpful for a user to consider all viable alternatives before delving deeply into one example.
Viewing multiple related examples together could provide a more complete view of class use cases, setup, use and teardown.
Our current study does not fully explore these advantages.
In the future, we plan to perform a larger-scale user study with programming tasks that require discovering new unfamiliar programming material.

We were surprised to see searchers use \systemname{} like Google search.
Users entered a query, selected the single most relevant question from the questions list, inspecting several examples, and moved on to refine their query or use code from the current examples.
\systemname{} can return multiple relevant questions related to a user's query for which answers can be viewed side by side.
For most users, however, we found that simultaneously examining answers to multiple questions would require a shift in current search habits.
Future versions of our tool can explore motivation techniques to help users select multiple answers and view a breadth of answers.

Based on our preliminary study, we could save programmers time by incorporating additional visual features into \systemname{}.
One user would have benefited from showing which code examples would compile in their IDE.
Additional features including votes, estimated runtime, and configurable parameters, could be integrated into the column representation to provide a more complete view of example usability.
Users may find it helpful to be able to highlight other examples similar to the lines they are currently previewing.
We are also interested in exploring how to incorporate question text into answer bars and how to cluster the answers based on the question they answer.

Ultimately, we look to apply the \systemname{} approach to resources beyond StackOverflow answers.
By examining document elements of blog postings and tutorials, we could build similar visual search results for the Google search index.
Column coloring could be modified to visually summarize structured learning materials like technical textbooks.
We plan to study how visualizations we produce could help learners in other domains discover more useful textual resources.

\subsection{Limitations}

We note that professional programmers only rarely adapt code snippets into their code.
As a result, our system may be of small utility to this user group.
The programmers that could use features like security and version checking are likely the ones who go to large effort to carefully understand the examples and their APIs prior to leveraging this code.
It seems that such users probably use non API-based or complex examples of more than a few lines only once a week or so.
