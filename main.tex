\documentclass[conference]{IEEEtran}

\usepackage{glossaries}
\usepackage[usenames,dvipsnames,svgnames,table]{xcolor}
\usepackage[pdftex,draft]{hyperref}
\usepackage[pdftex]{graphicx}
\usepackage{algpseudocode}
% \usepackage{enumitem}
\newglossaryentry{name}
{
    name=tutoroid,
    description={skill-specific text augmentation to existing tutorials that enable just-in-time training for a supplemental skill a tutorial glosses over.}
}
\newcommand\condcap[1]{\ifnum\ifhmode\spacefactor\else2000\fi>1000 \uppercase{#1}\else#1\fi}
\newcommand {\systemname}{\Glspl{name}}
\newcommand {\user}{Colin}
\newcommand {\userpro}{\condcap{h}e}
\newcommand {\userpos}{\condcap{h}is}

% Annotation macros
\newcommand {\andrew}[1]{{\color{red}\bf{AH: #1}\normalfont}}
\newcommand {\appachu}[1]{{\color{blue}\bf{CA: #1}\normalfont}}

\makeglossaries

\ifCLASSOPTIONcompsoc
  \usepackage[tight,normalsize,sf,SF]{subfigure}
\else
  \usepackage[tight,footnotesize]{subfigure}
\fi

% correct bad hyphenation here
\hyphenation{}

\begin{document}
\title{\Glspl{name}: Generating Context-Relevant, On-Demand Explanations and Demonstrations of Online Code}

% author names and affiliations
% use a multiple column layout for up to three different
% affiliations
\author{
\IEEEauthorblockN{
Andrew Head,
Codanda Appachu,
Marti A. Hearst,
Bj\"{o}rn Hartmann
}
\IEEEauthorblockA{
Computer Science Division, \\
U.C. Berkeley, Berkeley, CA 94720 \\
\{andrewhead, appachu, bjoern\}@eecs.berkeley.edu, hearst@berkeley.edu
}
}

% make the title area
\maketitle

\begin{abstract}

Programmers frequently turn to tutorials to learn frameworks and find new approaches to solving problems.
However, today's ``wild'' blog-based tutorials can contain pitfalls, errors, and underexplained concepts that require programmers to spend time looking up supplemental documentation.
We argue that the code-centered content of programming tutorials enables a new paradigm of \emph{on-demand, context-relevant documentation}.
\Glspl{name} are language- or library-specific routines for spinning on-demand, context-relevant explanations of unfamiliar code that users encounter in the wild.
\andrew{Clarify -- what do context-relevant and on-demand mean?}
\Glspl{name} parse code found in tutorials and generate in-situ natural language explanations based on user-authored patterns.
\Glspl{name} can be active anywhere on the web, enabling conceptual understanding with less effort and time.
\andrew{Concretify -- what does `conceptual understanding mean'?}
We show a family of \glspl{name} that generate multi-level descriptions, visualizations, context-relevant usage examples, and hints content introduced by other \glspl{name}' explanations.
We outline the interactions that \glspl{name} afford and the effort they demand from documentation developers.
Through a preliminary in-lab study and a collaboration with a documentation writer for a major API, we show that \glspl{name} produce helpful explanations, following up-front developer investment.
\andrew{`Wild code' includes code not in tutorials, such as on Github.}

\end{abstract}


% For peerreview papers, this IEEEtran command inserts a page break and
% creates the second title. It will be ignored for other modes.
\IEEEpeerreviewmaketitle
\pagestyle{plain} 
\section{Introduction}

End user programmers' develop by interleaving opportunistic information foraging with writing code~\cite{brandt_two_2009}~\cite{brandt_example-centric_2010}.
As online coverage of common APIs grows~\cite{parnin_measuring_2011}, programmers increasingly use \emph{crowd documentation} like StackOverflow and blog postings.
Such documentation can be written by coders who lack interest or expertise in writing usable documentation.
Both experienced programmers~\cite{duala-ekoko_asking_2012} and end-user programmers~\cite{dorn_lost_2013}\cite{dorn_learning_2010}\cite{rosson_everyday_2004} struggle to leverage web documentation to solve programming problems.
\andrew{Check \cite{dorn_learning_2010}\cite{rosson_everyday_2004} are relevant.}

Decades of research on technical communication provide best practices for writing usable documentation.
For learning how to use software, \emph{minimalist instruction}~\cite{carroll_nurnberg_1990} recommends enabling users to start immediatedly on realistic tasks, reducing the amount of reading and other passive activity, and supporting error recognition and recovery.
Farkas introduces \emph{layered documentation} as an extension to minimalist instruction that enables users from different backgrounds to benefit from the same documents by providing optional \emph{backup information} for tasks like error recognition and correction~\cite{farkas_layering_1998}.

Instructions for web scraping code may assumes a reader has some \emph{base competencies} in Python, callbacks, threads, and micro-languages like CSS selectors, regular expressions.
Given the \emph{diverse backgrounds of both the audience and authors}, it is impractical for the author to introduce all these base competencies in detail.
As a result, today's Q\&A's and blogs lack scaffolding for readers seeking clarification on unfamiliar or complex syntax.

To enable this scaffolded documentation, we propose \emph{\Glspl{name}}: on-demand, context-relevant descriptions of programming languages, commands and libraries that can run on arbitrary code found anywhere.
\andrew{Still not clear if \glspl{name} are explainers or explanations.}
Such explanations can describe syntax for languages like CSS selectors and regular expressions.
A \gls{name} for a language or command is programmed in several hundred lines of code to both \emph{parse} and \emph{explain} arbitrary input syntax.
They can be integrated as bookmarklets into a programmer's browser to provide just-in-time explanations of a language, command or API anywhere on the web.
We demonstrate how they can be used to generate text descriptions at multiple levels of detail.



Our contributions in this paper are as follows.
First, we demonstrate of a need for micro-explanations by showing underexplained, assumed \emph{base competencies} in popular StackOverflow answers.
Second, we propose an architecture for \glspl{name}, automatic natural language generators for describing programming languages, commands and APIs.
Third, we demonstrate the technical effort required to build \glspl{name} by building them for CSS selectors, command lines, and regular expressions.
Finally, we evaluate their utility through an informal in-lab study where we have users perform transfer tasks with tutorials augmented with \gls{name}-generated explanations.

\begin{figure}[!t]
\centering{
    \subfigure[Text augmentation explaining a CSS selector]{
        \framebox{\includegraphics[width=.4\textwidth]{figures/css_explanation}}
        \label{fig:css_explanation}
    }
}
\label{fig:tutorons}
\caption{Automatic, context-relevant textual and visual explanations of code generated by descriptive \glspl{name}.}
\end{figure}

\section{Definitions}

Throughout this paper, that we refer to the following terms to describe our system, its input and output:

A \emph{\gls{name}} is a routine on a web server with language-specific rules for detecting, parsing and explaining source code written on a web page.

A \emph{\gls{exp}} is the output of a \gls{name} for an explainable region of code. This can be a natural language explanation or a domain-specific demonstration of what the code does.

%%A \emph{code snippet} is a block of code on a page encompassed in a single DOM element --- e.g., a \texttt{<code>} or \texttt{<pre>} block --- that a programmer is taking effort to understand or reuse.

An \emph{explainable region} is a substring of an online code snippet that a \gls{name} can parse and explain.  For example, the term \texttt{div} in the JavaScript code \texttt{\$(`div').css(\{`display': `none'\})} is an explainable region for the CSS \gls{name}.

The \emph{\Glspl{name} addon} is a browser extension that queries \glspl{name} for \gls{exp} of code.  We use the addon to demonstrate the advantages of on-demand, context-relevant in-situ programming help.
\section{Related Work}

Our work is related to research from two themes.
In this paper, we compare \Glspl{name} to past work on the automatic explanation of code and demonstration of code.

\subsection{Automatic Explanation of Code}

Most closely related to our work is Webcrystal~\cite{chang_webcrystal_2012}, a tool that assists web authors with low-level programming tasks required to learn from and reuse examples.  
Webcrystal generates human-readable textual answers to user's ``how'' questions about how to recreate aspects of selected HTML elements as well as customized code snippets for the users to recreate the design.
Similarly to Webcrystal, we generate human-readable representations of code to help programmers reuse and learn from online examples.
In contrast to Webcrystal, we focus on describing short, embedded, cryptic languages like regular expressions and terminal commands instead of HTML, developing generalizable design patterns that can apply to generating explanations for other languages in the future.

In its technical approach, our work relates to recent efforts to generate natural language explanations of code and software engineering artifacts~\cite{sridhara_automatically_2011,burden_natural_2011,sridhara_towards_2010,kamimura_towards_2013,mcburney_automatic_2014,sridhara_generating_2011,haiduc_supporting_2010,moreno_automatic_2013}.
This body of work aims to reduce the amount of time developers spend reading poorly-documented code by automatically summarizing Java methods~\cite{sridhara_towards_2010}, blocks of code~\cite{sridhara_automatically_2011}, method parameters~\cite{sridhara_generating_2011}, and classes~\cite{moreno_automatic_2013}.
Like we do, Sridhara et al.\ consider how to detect explainable code prior to generating descriptions~\cite{sridhara_automatically_2011,sridhara_towards_2010}.
However, their code detection process emphasizes summarization of methods and blocks of code rather than locating instances of a language in potentially mixed-language code snippets.
We also distinguish ourselves from this past work by examining the code explanation problem from the perspective of several frequently used yet under-explained lanuages used in web-based programming help.
This enables us to test out and propose guidelines for detecting, parsing and explaining code by considering the syntax, purpose, and usage scenarios of each language.

\subsection{Automatic Demonstration of Code}

Software visualization has a long history of producing visual representations of code for many purposes~\cite{sorva_visual_2012}.
PythonTutor~\cite{guo_online_2013} is a recent programming visualization tool for CS education.
The tutor is embeddable within a web page and supports simultaneous viewing of program source code, program execution, visual representations of Python objects, and program output.
More recently, Ou et al.\ produced visualizations of pointer-based data structures in the heap~\cite{ou_interactive_2015}.
Beyond visualizations, code can be represented by discrete structures.
Dantoni et al.~\cite{dantoni_how_2015}\ demonstrate how to create counterexamples of system behaviors for helpful feedback for computer science students completing automata design problems.
We draw inspiration broadly from this past work in both visualization and example generation to automatically produce demonstrations of regular expresions and HTML\@.
We believe we are the first to consider building demonstrations from the perspective of writing design guidelines for micro-explanations of code found in online programming help.

%\section{Preliminary Study}

\andrew{Do the following studies. Numbers included below are optimistic placeholders.}
\andrew{A table would be great to summarize these results.}

StackOverflow is a popular online Q\&A site for programmers to ask and answer progrmaming problems, with more than 16 million answers to more than 9 million questions.

\begin{itemize}
\item Through an inspection of the top 200 most-viewed posts on StackOverflow, we find 10 \emph{micro-languages} that are not included as tags of the post, but that are used incidentally, including command lines, SQL, and regular expressions. \marti{unclear wording; also, do you want to mention wget explicitly here as a pointer forward?}
\item In addition, out of 100 top-rated posts related to Python that include regular expressions, we found that 50\% did not include any description of what the regular expression was doing, and 75\% provided a concise description or less.\marti{do we want a bullet for command lines?}
(We can do the same for CSS in jQuery selections.)
\end{itemize}

For users who find code with unexplained languages and syntax, we propose that we can reduce the cost of looking up supplemental information that explains these micro-languages by automatically generating relevant explanations of this code on the fly.

\section{Authors and Readers in the \Gls{name} Ecosystem}

In the current incarnation of \glspl{name}, an author of code documentation writes a \gls{name} in order to adaptively describe  code programmers find while they browse the web.
A \gls{name} is an engine for generating explanations or demonstrations of code for a specific language, accessible as a simple web API\@.
When queried with the text of a web page, a \gls{name} detects explainable regions of the language, parses them, and returns explanations for each region as formatted HTML that can be viewed in a tooltip.

\begin{figure}
%%\centering
    \includegraphics[width=\columnwidth]{figures/tutoron_ecosystem}
    \caption{%
    The \Glspl{name} ecosystem and information flow.
    A programmer browses for help in a web browser with the \Glspl{name} addon.
    When she visits a page, the addon queries a bank of \gls{name} servers for different programming languages.
    Each server detects explainable regions of code and produces \glspl{exp}, or explanations and demonstrations of this code.
    The programmer can then view these in-situ and on-demand in tooltip-style overlays by simpling selecting explainable regions of code with the mouse.
    }\label{fig:tutoron_ecosystem}
\end{figure}

\if 0
\begin{figure}
\centering
    \includegraphics[width=\columnwidth]{figures/explanation_pipeline}
    \caption{\Glspl{name} \emph{detect} relevant code snippets, \emph{parse} them, and then \emph{generate explanations}.  Here we show examples of the output of each stage of the pipeline for a \gls{name} that explains CSS selectors.}\label{fig:explanation_pipeline}
\end{figure}
\fi

By installing an addon for the browser, a programmer receives instant access to \glspl{exp}, or in-situ descriptions of code found while browsing, from \gls{name} servers (Figure~\ref{fig:browser_tutorons_markup}).
The addon queries existing \glspl{name} with the page source, receiving micro-explanations for all explainable regions of code.
After receiving explanations, an explanation will appear in a tooltip overlaid on the document directly beneath the source any time the programmer selects an explainable string of text.
The \Glspl{name} addon can query many explanation servers for multi-language support within the same page.
The information flow of the \Glspl{name} ecosystem is shown in Figure~\ref{fig:tutoron_ecosystem}.
We note that our approach could work in other settings, e.g., as a Wordpress plugin, which would shift the burden of installing software from user to tutorial author.

The addon uses a \emph{push} method for fetching explanations rather than a \emph{pull} method, requesting for a server to detect all explainable instances of a language for a webpage.
This choice enables all explanations to be generated in a single batch when the user first accesses the page, reducing the load time to milliseconds instead of seconds to access explanations.
Each server is queried in parallel once the document is ready.
The original webpage is instantly available to the programmer, and \glspl{exp} for each language become available as each \gls{name} processes the DOM\@.
Computational burden resides on the server, dependent on the implementation.
Client procedures consists of string matching to explainable regions detected followed by insertion of generated HTML into a tooltip, which will be linear in time to the number of explanations.

By requiring \gls{name} servers to detect explainable regions, we resolve the problem where a user's selection may not cover a complete statement of the syntax of a language.
Pre-computing these explainable text regions, we can allow users to select explainable code with \emph{fuzzy boundaries} (see Figure~\ref{fig:fuzzy_boundaries}).
While our qualitative evaluations have shown that this selection mechanism is easy to use, we recognize the future efforts could improve usability.
This includes highlighting explainable regions in colors corresponding to each \gls{name} to indicate what code programmers can gain clarification on, and showing tooltips on hover or right-click events.

\begin{figure}
    \centering
    \framebox{\includegraphics[width=\columnwidth]{figures/browser_tutorons_short}}\label{fig:browser_tutorons_markup}
    \caption{A user installs the \Glspl{name} addon.  \emph{(a)} Once she activates it, \emph{(b)} she can view automatically-generated, context-relevant explanations of code of supported languages in-situ while she browses programming help.}
\end{figure}

\begin{figure}
\centering{%
    \subfigure[The CSS selector.]{%
        \framebox{\includegraphics[width=.2\textwidth]{figures/selection_best}}\label{fig:selection_best}
    }
    \subfigure[The CSS selector with quotes.]{%
        \framebox{\includegraphics[width=.2\textwidth]{figures/selection_quotes}}\label{fig:selection_quotes}
    }
    \subfigure[A jQuery selection.]{%
        \framebox{\includegraphics[width=.2\textwidth]{figures/selection_jquery}}\label{fig:selection_jquery}
    }
    \subfigure[A sloppy selection.]{%
        \framebox{\includegraphics[width=.2\textwidth]{figures/selection_sloppy}}\label{fig:selection_sloppy}
    }
    \caption{%
    The \Glspl{name} addon activates explanations for code when a user selects text.
    Explainable regions of code are detected on an explanation server and returned to the browser, where the addon enables users to view explanations with fuzzy selections.
    This lets them view \glspl{exp} without knowing the exact syntactic boundaries of the code they want to have described.
    }\label{fig:fuzzy_boundaries}
}
\end{figure}

\section{How to Build a \Gls{name}}

A \gls{name} is a routine that \emph{detects}, \emph{parses}, and \emph{explains} code for a specific language in HTML documents.
We describe each processing stage with overarching strategies we have determined for finding relevant code in a page, parsing languages, and generating domain-appropriate explanations.
We couch our discussion in the implementation details of two tutorons we developed for CSS selectors and the \texttt{wget} command and micro-expanations we generate for regular expressions.

\subsection{Detection}
In the \emph{detection} stage, a \gls{name} should extract explainable regions from an HTML document using the language's lexicon and / or syntax.
This can consist of four steps.

First, the \gls{name} extracts blocks of code from HTML pages by selecting code-related elements.
Our current tutorons extract bloks of code from \texttt{<code>} elements, though other common code elements are \texttt{<pre>} and formatted \texttt{<div>} elements.

Second, it divides code blocks into candidate explainable regions based on language-dependent rules.
CSS selectors and regular expressions can be detected as string literals in parsed Python or JavaScript code, requiring an initial parsing stage to detect these candidate regions.
Command lines like \texttt{wget} often occupy a line of a script, meaning that candidate regions can be found by splitting code blocks into lines.

Third, it passes candidate regions through a language-specific syntax checker to determine if it follows the grammar.
Note that if this can be combined with the parsing stage of the \gls{name} processing pipeline.

Finally, a \gls{name} may reduce false positives by filtering candidates to those containing tokens highly representative of the language.
This is necessary when candidate regions are small and the terminals of a grammar accept large character classes.
While the string \texttt{``marti''} in a JavaScript program could represent a custom HTML element we in a selector, it's really more likely a string for some other purpose --- elements in a CSS selector more often than not have tag names defined in the HTML specification (e.g., \texttt{p}, \texttt{div}, or \texttt{img}).

\begin{changes}
In the future, many languages with similar syntax could be supported by a different \gls{name}.
When multiple explanations are available for the same explainable region, how do we disambiguate which explanation is the best one to use?
We propose a method of searching for `emblematic' patterns within a string that provide a confidence score that a region indeed belongs to a certain language.
For example, searching for the tokens `\textasciicircum' at the beginning of text or `\$' at the end of a text is highly emblematic of regular expressions, but not CSS selectors.
The substrings `div' and `table' are, likewise, highly likely to occur in a selector but not in a regular expression.
Through preliminary tests, we were able to demonstrate that with these confidence scores, we are capable of disambiguating between strings for regular \andrew{Finish this}
\end{changes}

\subsection{Parsing}

Detected code snippets are parsed into a parse tree.
We have found two methods of building parsers to be useful.
When it is necessary to recognize a wide range of symbols and their semantics, \gls{name} authors can modify existing parsers, introducing hooks for extracting results of parsing.
Though this process can be highly involved, requiring a \gls{name} author to have access to the source code and for him to understand how to extract the information and exit the application safely at the appropriate time.
Ultimately, given the involved rules in parsing command lines, this was the right step to take for robust parsing of \texttt{wget} command lines.

However, in other cases, it may be appropriate to develop a custom parser for the language that supports a relevant subset of the language.
For CSS selectors, we wrote a parser in 30 lines of ANTLR\footnote{\url{www.antlr.org}}, which offered several benefits.
First, we had complete control over the parse tree produced.
As our explanations relied on parse tree traversal from leaf element through its parents, we constructed the tree in the format we wanted to traverse.
Second, our parser generator could create tree visitors in both Java and Python, which enabled us to traverse the tree in Java to leverage our natural language software and build example HTML using more familiar Python-based DOM manipulation libraries.
Custom parsers may also be necessary when parser code is too difficult to instrument to extract a useful parse tree, or when source code for the parser is not available.

\subsection{Explanation}

During the final stage, \emph{explanation}, the \gls{name} traverses the parse tree to generate explanations and demonstrations of the code.
The implementation details of this stage is specific to the structure of the parse tree and the ideal representation of the code.
In the next section, we describe techniques that we hope provide inspiration for future language-specific \glspl{exp}.

\section{Examples of \Glspl{name}}

\subsection{Traversing syntax trees to describe code}

We demonstrate an example of walking a syntax tree to generate a natural language description of a CSS selector in Figure~\ref{alg:css_traversal}.

\begin{figure}
\begin{algorithmic}

\Function{visit\_node}{node}
    \State clause = plural(node.element)
    \If{node has id}
        \State clause += (` with ID ' + node.id)
    \ElsIf{node has class}
        \State clause += (` of class ' + node.class)
    \EndIf
    \If{node has child}
        \State ch\_clause = visit\_node(node.child)
        \State clause = ch\_clause + `from' + clause
    \EndIf
    \State \Return clause
\EndFunction

\State
\Function{main}{parse\_tree}
    \State root\_clause = \Call{visit\_node}{parse\_tree.root}
    \State print ``This selector chooses '' + root\_clause
\EndFunction

\end{algorithmic}
\label{alg:css_traversal}
\caption{Procedure for generating text description of a CSS selector by traversing the parse tree.
\andrew{Note that a cheap shot here might be, how do we describe a node with both a class and ID?
The lame answer is that we currently don't.}
}
\end{figure}

\subsection{Mining and exposing hierarchy}

\begin{figure}
\centering{
    \framebox{\includegraphics[width=.4\textwidth]{figures/so_wget_explanation}}
    \label{fig:so_wget_explanation}
    \caption{Expandable textual augmentation describing overview of command, followed by a fine-grained detailing of its arguments.}
}
\end{figure}

Using StackOverflow, developers can mine combinations of parameters that mean more than the sum of individual parameters.
For example, for \texttt{wget}, the combination of the \texttt{-r}, \texttt{-l<level>}, \texttt{-A<ext>}, and \texttt{-e robots=off} tags suggest that a user is about to scrape a destination URL recursively to depth level \texttt{l} for files of type \texttt{A}, without following the recommended etiquette for crawler \texttt{robots}.
\bjoern{This is really cool — does the technique apply generally to other types of tutorons?}
Through a preliminary effort with \texttt{wget}, we showed that one can find reasonable combinations of arguments that should be described together by scraping and observing frequent combinations of arguments from commands scraped from StackOverflow (Figure~\ref{fig:wget_arguments}).

Using this procedure, we can create explanations like those in (Figure~\ref{fig:so_wget_explanation}).
An initial description may just contain an overview of what the command does as a whole, inferred from the combination of arguments.
However, asking for information can provide a detailed breakdown of how each option affects the execution of the command.

\begin{figure}
\centering{
    \includegraphics[width=.3\textwidth]{figures/wget_arguments}
}
\caption{\texttt{wget} arguments that frequently co-occur in uses of \texttt{wget} in StackOverflow questions. \andrew{A table would may be more readable here.}}
\label{fig:wget_arguments}
\end{figure}

\subsection{Generating examples}

\bjoern{Wait, there are so many different examples flying by right now.
Would all tutorons use the  techniques in sub-sections A-D here? Or each tutoron just one? I’m a bit confused about the higher-level argument.}
\bjoern{I think instead of showing a bunch of point examples that are all really  different from each other, we need to think about commonalities between them.
Otherwise reviewers may wonder why there’s a regular expression sub-project described here that doesn’t connect to other examples.}
We expect that similar approaches could be used to automically generate just-in-time exercises related to content of code in tutorials users are viewing. 
\andrew{Some reference to related work in exercise generation here would be appropriate.}

Long regular expressions are cryptic even for expert readers to understand.
We expect that in some contexts, providing readable examples of what strings will satisfy and fail to satisfy these patterns could help with comprehension of the regular expressions.

For the sake of showing that this concept would work, we determined guidelines for producing example strings that match a regular expression pattern.
Although there is an unbound number of strings that satisfy regular expressions that have repetitions, we want to limit the number of results to a readable, representative group.
We arbitrarily choose an ideal of 3-4 positive and 3-4 negative examples, though provide users the ability to request additional examples.

Cursorily, the process of generating relevant examples is as follows.
We start by producing an \emph{urtext}, a base version that satisfies the regular expression.
For each repeated group (not just a character) (* or +), we generate it exactly once to make sure that the group appears in the original best example.
We print any literal exactly as it appears in the pattern.
We choose on representative of each branch.
Whenever we have a repeated sequence of alphanumeric characters (repeated by + or *), we choose a dictionary word 5 letters in length and append 2 numbers.
If there are any symbols in the character class for this word, we choose a random sample of 2 of these symbols.
This forms our \emph{urtext}, which is the first positive example.

We view each regular expression as a graph of \emph{branches}, \emph{choices}, \emph{repetitions}, and \emph{literals}.
To create an alternative correct answer, we perform each just one operation to one node on top of the urtext:
\begin{itemize}
\item Flip a branch to choose a different child
\item Choose an alternative dictionary word and sampling of numbers and symbols for word choices
\item For + repetitions, toggle 1 repetition to 2.  For * repetitions, toggle 1 repetition to 0.
\end{itemize}

We generate alternative correct answers by performing one operation at each node, holding the rest of the tree at its urtext form.
We suspect that altering only one part of the string while holding the rest constant will improve reader's ability to detect what change is acceptable for the string to satisfy the pattern.
We choose the urtext and randomly sample 3 valid alternatives for the correct examples of a regular expression.
Similarly, we alter one character or repetition at a time for each node to generate failure cases.

In the future, we would like to implement in-place editors of regular expressions that enables real-time updating of these examples and checking user-defined test patterns.

\section{Evaluation}

\subsection{Procedure}

We conducted a qualitative usability study to understand how \Glspl{name}-generated \glspl{exp} affected programmers' ability to perform code modification tasks with online example code.
We recruited 9 programmers from university listservs for undergraduate and graduate students in computer science and information science.
All participants had at least two years of programming experience, with a median of 4 years of experience in their longest-used language.  Participants were asked to explain their actions via think-aloud method which was audio-recorded, and their actions were screen-captured and analyzed post-hoc. 

Each participant attempted 8 code modification tasks (plus 2 practice tasks) using two different languages: CSS selectors and \texttt{wget}.
For each language the 4 coding tasks increased in difficulty with the fourth being especially tricky.
We created snippets consisting of a block of code and optionally  explanatory text and comments for each task, based on  existing online programming help.

Each code modification task consisted of the following steps:
\begin{enumerate}
\item Read a task --- e.g., \emph{write a CSS selector that selects only elements of class \texttt{myInput}.}
\item View a snippet with some clue about the task.
E.g., in Figure~\ref{fig:study_snippet}, the participant was shown a snippet containing CSS selectors that choose elements of class \emph{myCheckbox} without explaining the syntax.
\item Write and test the code in a sandbox we provided.
\end{enumerate}
After reading the snippet, to solve the task, participants could use any resources they wanted to, including searching the web.
Participants were told they would have 5 minutes for each task; when participants had extra time, if  they had not completed the task, we often asked them to continue so we could observe their problem-solving process. When in the micro-explanation condition participants were asked to find and view all \glspl{exp} for the source code after reading the snippet.

\begin{figure}
\centering
\framebox{\includegraphics[width=\columnwidth]{figures/study_snippet}}
\caption{An example snippet shown as a clue for a code modification task, accompanied by a micro-explanation that a participant in our study would have viewed if they were instructed to do so in this task.}
\label{fig:study_snippet}
\end{figure}

Participants viewed \glspl{exp} for alternating tasks so we could observe differences in how they approached the tasks with and without micro-explanations; exposure to micro-explanations was counter-balanced across participants. 

%%This ordering was counterbalanced across participants so that we could see all tasks performed both with and without \gls{name}-generated \glspl{exp}.



\subsection{Results}

Our primary goal in observing participants was to determine if \gls{name}-generated \glspl{exp} were helpful  during code modification tasks and reduced the need to reference additional documentation.
In the discussion below, we refer to individual participants as $P{1-9}$.

\subsubsection{ \Glspl{exp} Help Programmers Modify Code Without Using Other Documentation}

\begin{figure}
\centering
\includegraphics[width=\columnwidth]{figures/doc_accesses}
\caption{
Tasks for which participants sought additional documentation beyond the snippet. In white rows, participants used \glspl{exp}; in gray rows, they did not.
A cell is marked with the letter `D' if a participant accessed external documentation during that task.
}
\label{fig:doc_accesses}
\end{figure}

When using the micro-explanations, 34 out of 36 tasks, or 94\% did not require external documentation.  However, for those tasks where micro-explanations were not available, 63\% of the time participants did indeed access external resources in order to complete the task.  The wget tasks especially required external help, with man pages and other resources being accessed in 89\% of cases.  
A detailed breakdown of external document usage is shown in Figure \ref{fig:doc_accesses}. 


%%In total, out of 72  tasks, participants sought out additional documentation a total of 25 times (Figure~\ref{fig:doc_accesses}).
%%In all wget tasks and in the fourth CSS selector task, a majority of participants without \glspl{name} viewed documentation at some point while trying to find a solution to the problem.
%%In the all 4 CSS selector tasks and the first 3 wget tasks, no participants with access to the \glspl{exp} sought additional documentation in order to solve the problem.

%Only in the final wget task did any participants with access to \glspl{name} seek documentation.


These preliminary results suggest that the micro-explanations are effective at reducing the need to switch contexts to find relevant programming help  for completing some programming tasks. 
The \gls{exp} aided the programmers in several ways:

{\bf Reducing need to access external documentation.}
Participants were  able to identify which flags had to be removed from wget commands without consulting external documentation, despite not having used the wget before ($P4$).
For others, the \gls{exp} affirmed a guess that the participant already had about how to construct a CSS selector ($P1$).

{\bf Context-relevant pattern matching.}
One participant noted that the \glspl{exp} helped  to map flags for the wget command line to the higher-level intent of the task ($P4$). 
For the most complex CSS selector task (see Figure \ref{fig:study_snippet}), two participants noted
that the example HTML shown in the \gls{exp} 
provided a pattern of what fields needed to be changed  from the selector in the snippet to capture the element and ID required by the task prompt ($P2$, $P4$).

{\bf Learning programming concepts.}
For another participant with little previous experience with CSS selectors, a \gls{exp} in  the first task provided him with the knowledge needed to write the selector for the next two tasks, one task for which he was not allowed to view \glspl{exp} ($P5$).

\subsubsection{Programmers Without \Gls{exp} Searched External Documentation}

Some of the difficulties participants faced in the no-\gls{exp} condition  highlight the benefits of in-situ help.
Some participants had difficulty searching for programming help on a web search engine (Google) and using the seach results.
One participant could not express the symbols `\texttt{\^{}=}' as a query term when searching for what  this pattern signifies in CSS selectors.
Her follow-up queries with English language query terms  yielded search results that were not relevant ($P3$).

Participants also had difficulty navigating conventional forms of programming help.
When looking for the \texttt{-r} flag in the man page for wget ($P2$, $P4$), participants found that a page-internal  search for the flag yielded so many matches that it was difficult to find the desired definition.
The description of the \texttt{-N} timestamp flag that was relevant to the code modification task was not directly adjacent to where the flag was introduced, causing one participant to overlook this information, even though it was only 10 lines away ($P3$).

These results underscore the usefulness of context-relevant explanations located within the tutorial text itself.

\subsubsection{Opportunities for Improvement}
In those cases where the \gls{exp} did not aid programmers, there were a few primary causes.

{\bf No visual affordances.} We did not place  visible cues showing where explanations were available.
Programmers may fail to find micro-explanations since they have to be explicitly invoked through a right-click.
Although $P2$ eventually found a micro-explanation that led him to writing a successful CSS selector for the most difficult CSS task, he had to click around the snippet to find it, and did so only upon being reminded by the experimenter that he had not viewed all the \glspl{exp} in the snippet ($P2$).  We plan to experiment with providing visual affordances for the presence of \glspl{exp} in future.


{\bf Selection region ambiguity.}
The leniency of our algorithm for matching a text selection to an explained code fragment caused confusion for programmers about which fragments would be explained on each page ($P1$, $P2$, $P5$).
For the same reason, explanations generated did not always match the fragment selected ($P1$, $P3$, $P5$).
For example, one participant selected the text \texttt{<p>} for which no explanation was generated by the explanation server, because no CSS selector starts with a less-than sign.
However, our selection matching algorithm matched this string to the \texttt{p.mainPageMeters} selector for which an explanation \emph{had} been generated by the server.  As a result, this participant viewed an irrelevant explanation for the code fragment he was viewing ($P5$).  More work is required to determine the right balance of fuzziness for the matching algorithm, with perhaps a drop-down menu of choices for alternative matches.

{\bf Incomplete explanations.} The  \gls{exp} text may not include enough detail to help programmers  develop adequate mental models of unfamiliar material. For instance,
even after completing all 4 CSS selector tasks, $P5$ appeared to believe that CSS selectors were HTML elements themselves, rather than labels that could fetch them, perhaps confused by the example HTML produced in each \gls{exp} ($P5$).  That said, the idea of \gls{exp} can be expanded by adding links to fuller tutorial material.
\section{Discussion}

StackOverflow and Google are already usable, functional interfaces for finding programming solutions.
however, \systemname{}'s unique strength arises when there are multiple ways to solve a problem and many online resources describing how to do it.
A breadth-first consideration could be helpful for a user to consider all viable alternatives before delving deeply into one example.
Viewing multiple related examples together could provide a more complete view of class use cases, setup, use and teardown.
Our current study does not fully explore these advantages.
In the future, we plan to perform a larger-scale user study with programming tasks that require discovering new unfamiliar programming material.

We were surprised to see searchers use \systemname{} like Google search.
Users entered a query, selected the single most relevant question from the questions list, inspecting several examples, and moved on to refine their query or use code from the current examples.
\systemname{} can return multiple relevant questions related to a user's query for which answers can be viewed side by side.
For most users, however, we found that simultaneously examining answers to multiple questions would require a shift in current search habits.
Future versions of our tool can explore motivation techniques to help users select multiple answers and view a breadth of answers.

Based on our preliminary study, we could save programmers time by incorporating additional visual features into \systemname{}.
One user would have benefited from showing which code examples would compile in their IDE.
Additional features including votes, estimated runtime, and configurable parameters, could be integrated into the column representation to provide a more complete view of example usability.
Users may find it helpful to be able to highlight other examples similar to the lines they are currently previewing.
We are also interested in exploring how to incorporate question text into answer bars and how to cluster the answers based on the question they answer.

Ultimately, we look to apply the \systemname{} approach to resources beyond StackOverflow answers.
By examining document elements of blog postings and tutorials, we could build similar visual search results for the Google search index.
Column coloring could be modified to visually summarize structured learning materials like technical textbooks.
We plan to study how visualizations we produce could help learners in other domains discover more useful textual resources.
\section{Conclusion}

\begin{abstract}

Programmers frequently turn to tutorials to learn frameworks and find new approaches to solving problems.
However, today's ``wild'' blog-based tutorials can contain pitfalls, errors, and underexplained concepts that require programmers to spend time looking up supplemental documentation.
We argue that the code-centered content of programming tutorials enables a new paradigm of \emph{on-demand, context-relevant documentation}.
\Glspl{name} are language- or library-specific routines for spinning on-demand, context-relevant explanations of unfamiliar code that users encounter in the wild.
\andrew{Clarify -- what do context-relevant and on-demand mean?}
\Glspl{name} parse code found in tutorials and generate in-situ natural language explanations based on user-authored patterns.
\Glspl{name} can be active anywhere on the web, enabling conceptual understanding with less effort and time.
\andrew{Concretify -- what does `conceptual understanding mean'?}
We show a family of \glspl{name} that generate multi-level descriptions, visualizations, context-relevant usage examples, and hints content introduced by other \glspl{name}' explanations.
We outline the interactions that \glspl{name} afford and the effort they demand from documentation developers.
Through a preliminary in-lab study and a collaboration with a documentation writer for a major API, we show that \glspl{name} produce helpful explanations, following up-front developer investment.
\andrew{`Wild code' includes code not in tutorials, such as on Github.}

\end{abstract}

\andrew{TODO make different than abstract}


\section*{Acknowledgments}
This work was supported by NSF IIS 1149799.

% trigger a \newpage just before the given reference
% number - used to balance the columns on the last page
% adjust value as needed - may need to be readjusted if
% the document is modified later
%\IEEEtriggeratref{8}
% The "triggered" command can be changed if desired:
%\IEEEtriggercmd{\enlargethispage{-5in}}

% references section
\bibliographystyle{IEEEtran}
\bibliography{references}

\end{document}
