\section{Introduction}

The proliferation of programming resources on the web has enabled the growth of an enormous programmer community.
Relevant content for solving programming problems resides across an abundance of tutorials, MOOCS, programming examples, Q\&A sites and technical blogs.
These resources are popular for programmers of varying skill levels and fill in the inadequacies of traditional documentation and formal references.
Today's coding workflow contrasts starkly from that of a decade ago.
Programmers retrieve web resources and engage in self-led learning during each iteration of code implementation and foraging~\cite{brandt_two_2009}~\cite{brandt_example-centric_2010}.

A search for a solution to a programming problem begins with a Google search, leading to blog posts, tutorials, reference documentation, forums, and online Q\&As.
Searchers assess the relevance of online examples prior to extensive manual inspection of documents in multiple ways.
First, they rely on their search engine to rank results such that those at the top of the page will provide a relevant solution.
Second, they examine the textual excerpts returned with each result for additional context.
Third, for online Q\&A like StackOverflow, they assess which examples will be most helpful based on votes and user reputation.

Whether a programmer can successfully understand and integrate code examples into his code depends on many factors.
Code examples should build upon familiar libraries and explanatory text should use known terms or provide a definition of them.
If they contain errors, the programmer should be able to find related examples with the same classes and functions to check his work.
The programmer may already know whether he is looking for long textual responses or short code snippets. 
Modern search engines should offer a way for a user to quickly and effectively assess which examples contain a promising amount of explanation, usable libraries and terminology.

By conducting a preliminary user study with 5 programmers, we examine users' obstacles as they search for solutions to unfamiliar programming problems.
We identify 3 challenges users face using general search interfaces.
First, users search for results of particular format, like tutorials or code snippets.
Second, they engage in backtracking after discovering that previously abandoned examples contain well-supported libraries that they were trying to avoid.
Third, they maintain a bloated history through a trail of tabs in the browser of examples they may want to revisit later.

We develop \systemname{}, a search user interface that provides solutions to these problems through visual features.
Our tool represents text and code of code examples in colored columns to allow easy assessment of content distribution for each search result.
It generates charts of commonly used classes and coding concepts to help users with early detection of frequently used dependencies used for solving a problem.
A preview panel provides a way to rapidly skim through the content of multiple examples without requiring navigation through tabs or links.
A snippet bank enables users to maintain an in-tab history of promising code segments and explanations and to compare two examples side-by-side.
\systemname{} leverages the StackExchange API to gain access to answers to over 700,000 questions on StackOverflow, providing an overview of dozens of relevant answers with each query.

This work concludes with a 4-subject user study to evaluate the usefulness of our tool.
We determine that \systemname{} enables tasks like finding the most commonly used classes across examples where a baseline search interface does not.
Participants find it useful to be able to pinpoint structural code elements like functions, viewing them across many examples at once.
They also see value using the system in the early stages of learning when required to use libraries in unfamiliar programming tasks.
We finish with a discussion of future directions, focusing on the challenges of encouraging users to use a breadth-first approach in when finding programming solutions.