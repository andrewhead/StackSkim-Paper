\section{Introduction}

Major arguments here:
\begin{itemize}
\item Programmers rely on online tutorials to solve problems
\item Users encounter errors, misunderstandings, and shortages of information when working with these materials
\item We present a system for collaborative markup of parts of tutorials that cause these problems
\item A formative study describes the places in existing tutorials where users want more information
\item We design and implement \Glspl{name} for both one-off edits and across-the-web suggestions
\item We evaluate user interactions when authoring and using \Glspl{name}
\end{itemize}

We are interested in exploring the feasibility of creating \glspl{name} that are both highly context-sensitive and generalizable.
In addition, we hope to make first strides in identifying how \glspl{name} can be built from existing online documentation, to better link together existing web documentation in a user-consumable form from its disparate locations.

\andrew{Probably want to keep the following in some form.}
Programmers retrieve web resources and engage in self-led learning during each iteration of code implementation and foraging~\cite{brandt_two_2009}~\cite{brandt_example-centric_2010}.

\andrew{We can include a mention that we replicate the spirit of modern-day templating engines when providing the ability to generate descriptions within the DOM.  Can we work this into the design?}
\andrew{We can also phrase this paper as techniques for automatic augmentation of tutorials.}
