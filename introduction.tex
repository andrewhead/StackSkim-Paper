\section{Introduction}

Major arguments here:
\begin{itemize}
\item Programmers rely on online tutorials to solve problems
\item Users encounter errors, misunderstandings, and shortages of information when working with these materials
\item We present a system for collaborative markup of parts of tutorials that cause these problems
\item A formative study describes the places in existing tutorials where users want more information
\item We design and implement \Glspl{name} for both one-off edits and across-the-web suggestions
\item We evaluate user interactions when authoring and using \Glspl{name}
\end{itemize}

We are interested in exploring the feasibility of creating \glspl{name} that are both highly context-sensitive and generalizable.
In addition, we hope to make first strides in identifying how \glspl{name} can be built from existing online documentation, to better link together existing web documentation in a user-consumable form from its disparate locations.

We attempt to fill the gap between canned descriptions of functions in the API and IDE auto-complete and the manually crafted descriptions of concepts in good tutorials.
\Glspl{name} can provide the means for automatically generating context-relevant code descriptions.

\andrew{Probably want to keep the following in some form.}
Programmers retrieve web resources and engage in self-led learning during each iteration of code implementation and foraging~\cite{brandt_two_2009}~\cite{brandt_example-centric_2010}.

\andrew{We can include a mention that we replicate the spirit of modern-day templating engines when providing the ability to generate descriptions within the DOM.  Can we work this into the design?}
\andrew{We can also phrase this paper as techniques for automatic augmentation of tutorials.}

\subsection{Background}

\subsubsection{Writing Good Technical Documentation}

Minimal instruction theory provides guidelines for developing usable technical documentation~\cite{carroll_nurnberg_1990}.
There are three main insights of the minimalist approach:
learners are allowed to start immediatedly on meaningfully realistic tasks,
the amount of reading and other passive activity in training is reduced,
and errors and error recovery are presented in a way to make them less traumatic and more pedagogically productive.
Interstingly, today we find ourselves at a where that much of the typical programming documentation is no longer developed by professionals.
Programmers increasingly use \emph{crowd documentation}~\cite{parnin_measuring_2011} found on the web that may be written by hobbyists without formal training writing usable documentation and who do not have QA resources to cross-check documents' quality.
\andrew{Is this the right reference for the term crowd documentation?}

When it is not possible to produce minimal instruction that has been iteratively tested and tailored to its audience, Farkas recommends \emph{layered documentation}~\cite{farkas_layering_1998}.
Layered documentation allows users to access more \emph{backup information} for tasks like error recognition and correction, and enables the same documentation to be used by readers from different backgrounds.
We position our work with the core belief that minimal instruction is a worthwhile but likely unattainable standard for online programming tutorials.
Given the diverse audiences of programming tutorials and the lack of documentation expertise of their authors, minimal instruction seems impossible to achieve.
We therefore adapt Farkas's advice with a technique for adding interactive layering to existing tutorial documentation on the web.
Through this, we approach the aims of minimal instruction: improved transfer of tutorial skills to personal tasks, less web search for discovering background knowledge, and faster error recovery.

Eiriksdottir \& Catrambone~\cite{eiriksdottir_procedural_2011} detail three types of instructions --- procedures, principles, and examples --- and their pedagogical aims.
Detailed procedures and relevant examples are likely to ease users' initial task performance.
More abstract procedures promote good learning and transfer.
We believe that \glspl{name} can improve relevance of more general procedures by providing just-in-time description of configurable parameters of APIs.
Furthermore, they can be used to describe relevant principles and interactive examples to aid transfer beyond the current task.

\subsubsection{Measuring tutorial quality}

Lafreniere et al.~\cite{lafreniere_understanding_2013} determined the type of activites that users engage in through comment-based discussion following the body of tutorials.
\andrew{What were the activities, how do people learn from tutorials, what are their weaknesses?}
In the programming domain, Parnin \& Treude~\cite{parnin_measuring_2011} find that 87.4\% of the methods of the jQuery API are described by a blog post in the first 10 web search results.
Of these blog posts, about half were tutorials.

In a survey of 154 Photoshop tutorials, Lount \& Bunt~\cite{lount_characterizing_2014} harvested tutorials from an application-centered community, tutorial aggregator, tutorial factory, and popular tutorials via the CUTS tecnique~\cite{fourney_characterizing_2011}.
Among other findings, Lount \& Bunt describe strengths and weaknesses of the corpus as a whole.
Almost all tutorials included source files (90.3\%), the initial image (91.6\%), final image (96.7\%), at least one image per step (84.4\%), and references to past steps when repeated steps were used (89.6\%).
However, 40.7\% of tutorials contained no attempts to address potential errors, 83.1\% omitted any version information, and only 1 of every 20 steps had any explanation.
Mechanisms to improve such tutorials' coverage of potential efforts and explanation of tools used could improve the quality of the average online tutorial.


\andrew{Also possibly relevant:~\cite{ames_just_2001}, for when and how to intervene with instructions.}
