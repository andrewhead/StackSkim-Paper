\begin{abstract}

Programmers frequently turn to tutorials to learn frameworks and find new approaches to solving problems.
However, today's ``wild'' blog-based tutorials can contain pitfalls, errors, and underexplained concepts that require programmers to spend time looking up external documentation and debugging errors others have already encountered.
While the value of community cultivation and maintenance of knowledge bases and annotations on the web is well-explored, we suggest that \emph{code-centered markup} of the web enables new ways of detecting pitfalls, generating custom help, and sharing information.
In this paper, we describe \Glspl{name} framework, a system for community-led augmentation and maintenance of programming tutorial documentation.
By implementing a variety of \Gls{name} generators, we outline a design space of tools for code-based online tutorial augmentations, from Simple \Glspl{name} for trivial in-browser code annotation, to Power \Glspl{name} for automatically generating explanations of code fragments found anywhere on the web.
Through an in-lab qualitative study, we demonstrate that community-sourced \Glspl{name} can be fast to create and enable significant time savings for tutorial readers.
With a quantitative analysis of Power \Glspl{name}, we show that automatic augmentations can reach across dozens of tutorials for key programming tasks.
\andrew{Careful as to whether we define \Glspl{name} as a framework, as individual augmentations, or something else.
Be consistent!}

\end{abstract}
