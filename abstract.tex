\begin{abstract}

Programmers frequently turn to tutorials to learn frameworks and find new approaches to solving problems.
However, today's ``wild'' blog-based tutorials can contain pitfalls, errors, and underexplained concepts that require programmers to spend time looking up general-purpose supplemental documentation.
We argue that the code-centered content of programming tutorials enables a new paradigm of \emph{on-demand, context-relevant documentation}.
In this paper, we introduce such a paradigm, which we call \Glspl{name}.
\Glspl{name} are grammar- or library-specific routines for spinning context-relevant, on-demand explanations of unfamiliar code users encounter in the wild.
\Glspl{name} can be active anywhere on the web, enabling conceptual understanding with less effort and time.
We show a family of \glspl{name} that generate multi-level descriptions, visualizations, context-relevant usage examples, and hints about other \glspl{name}.
We outline the interactions that \glspl{name} affords and the effort they demand from documentation developers.
Through a preliminary in-lab study and a collaboration with a documentation writer for a major API, we show that \glspl{name} produce helpful explanations, following up-front developer investment.

\end{abstract}
