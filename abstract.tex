\begin{abstract}
Programmers frequently turn to the web to solve problems and find example code.
For the sake of brevity, the snippets in online instructions often gloss over the syntax of languages like CSS selectors and unix commands.
Programmers must compensate by consulting external documentation.
\if 0
This leaves programmers on their own to resolve errors and modify code by using external documentation. \marti{Alternative: Programmers must compensate by consulting external documentation.}
\fi
In this paper, we propose language-specific routines called \Glspl{name} that automatically generate \emph{context-relevant, on-demand} \glspl{exp} of code.
A \Gls{name} detects explainable code in a web page, parses it, and generates in-situ natural language explanations and demonstrations of code.
We build \Glspl{name} for CSS selectors, regular expressions and the Unix command ``wget''.
We demonstrate techniques for generating natural language explanations through template instantiation, synthesizing code demonstrations by parse tree traversal, and building compound explanations of co-occurring options.
Through a qualitative study, we show that \Gls{name}-generated explanations can reduce the need for reference documentation in code modification tasks.
\if 0 \andrew{Marti wants us to mention insights from the usability study here.}\fi
\end{abstract}
