\begin{abstract}

Programmers frequently turn to tutorials to learn frameworks and find new approaches to solving problems.
However, today's ``wild'' blog-based tutorials can contain pitfalls, errors, and underexplained concepts that require programmers to spend time looking up supplemental documentation.
We argue that the code-centered content of programming tutorials enables a new paradigm of \emph{on-demand, context-relevant documentation}.
\Glspl{name} are language- or library-specific routines for spinning on-demand, context-relevant explanations of unfamiliar code that users encounter in the wild.
\andrew{Clarify -- what do context-relevant and on-demand mean?}
\Glspl{name} parse code found in tutorials and generate in-situ natural language explanations based on user-authored patterns.
\Glspl{name} can be active anywhere on the web, enabling conceptual understanding with less effort and time.
\andrew{Concretify -- what does `conceptual understanding mean'?}
We show a family of \glspl{name} that generate multi-level descriptions, visualizations, context-relevant usage examples, and hints content introduced by other \glspl{name}' explanations.
We outline the interactions that \glspl{name} afford and the effort they demand from documentation developers.
Through a preliminary in-lab study and a collaboration with a documentation writer for a major API, we show that \glspl{name} produce helpful explanations, following up-front developer investment.
\andrew{`Wild code' includes code not in tutorials, such as on Github.}

\end{abstract}
