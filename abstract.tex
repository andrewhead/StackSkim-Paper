\begin{abstract}
Programmers frequently turn to the web to solve problems and find example code.
Online instructions gloss over the syntax and purpose of snippets in supplemental languages like CSS selectors and command lines for the sake of brevity.
This leaves programmers on their own to resolve errors and modify code by using external documentation.
In this paper, we propose language-specific routines called \Glspl{name} that automatically generate \emph{on-demand, context-relevant} \glspl{exp} of code.
A \Gls{name} detects explainable code in webpages, parses it, and generates in-situ natural language explanations and demonstrations of code.
By building \glspl{name} for CSS selectors, regular expressions and wget, we demonstrate techniques for generating natural language explanations through template instantiation, synthesizing code demonstrations by parse tree traversal, and building compound explanations of co-occurring options.
Through a qualitative study, we show that \gls{name}-generated explanations can reduce the need for reference documentation in code modification tasks for CSS selectors and \emph{wget} commands.
\end{abstract}