\section{Preliminary Study}

To discover programmers' search habits when looking for code examples in unfamiliar domains, we conducted a preliminary 5-subject study.
We refer to the subjects as \emph{P1 - P5}.
Of the subjects (2 female), 3 studied engineering or information science (\emph{P2, P3, P5}), 1 was a full-time industrial engineer (\emph{P4}), and 1 had no technical background outside of a semester of introductory programming (\emph{P1}).
All had written code in the last month.

We conducted semi-structured interviews with all participants (\emph{P1 - P5}) to understand how they solved programming problems using online resources.
For four participants (\emph{P2 - P5}), we provided a task in a language they had used previously but claimed they were \emph{not} comfortable with, and observed how they looked for solutions online.

During interviews, we asked questions to understand how users found answers in unfamiliar domains of programming:
\begin{enumerate}[noitemsep]
\item What sources do you use for locating code examples?
\item What are the features of a useful code example?
\item How many examples do you typically have to look at before discovering one that is useful?
\item What sources help you to learn the most while you are programming?
\end{enumerate}

For the coding task, we asked participants to use a language they had used in the last 6 months which they felt uncomfortable using.
We assigned them simple tasks in these languages, but ones that they would be likely to forget how to do had they not performed the task recently.
For example, we provided \emph{P4} with a task to write a program in C that printed time to the screen with a period of 1 second between each print.
\emph{P2} and \emph{P3} were asked to perform string concatenation in C++ and write the result to the screen.
\emph{P5} was asked to print a random number between 1 and 1,000,000 in bash.

\subsection{Findings about Search Behavior}

We observed several trends that informed the design of \systemname{} as a search tool:

\emph{Programmers prefer different sources at different learning stages.}
The novice programmer (\emph{P1}) claimed that when searching for answers to programming problems in her classes, she most frequently referenced class notes.
After this, she relied on general web search to search for exact answers to her problems.
For her current online course in programming, she would look for online blogs of past students who had encountered and solved the same problem.
Programmers more used to solving open-ended problems with code \emph{(P2, P3)} stated that they frequently used answers they found on StackOverflow.
One user \emph{(P5)} claimed he regularly used textbooks when looking up tiny examples related to his problems.
However, he clarified that textbook examples were often insufficient to provide comprehensive answers to the larger system-level problems he encountered at his IT internship.

\emph{Users revisit abandoned search results when later searches are not successful.}
Two programmers \emph{(P3, P5)} revisited search results they had previously abandoned while searching for examples during coding tasks.
They abandoned early results when they seemed complex or unfamiliar.
However, if later results also lacked promise or used the same unknown libraries,  participants sometimes backtracked.
A trail of tabs kept history of the pages they had visited.
Participants accessed old examples by looking back through old tabs.

\emph{Finding a usable solution can require iterative search.}
Two users (\emph{U4, U5}) refined their queries during the coding task to improve their search results.
One user (\emph{U5}) performed 10 separate queries through Google to discover how to print a random number using Bash.
Many queries were variations on a theme, for example \emph{bash masking}, \emph{bash command masking}, and \emph{bash command escape}.
Queries also targetted distinct steps of the full problem, including generating a random number, saving the value, and retrieving a larger random value than the default range.

We noticed several trends that inspire future additions to \systemname{}.

\emph{Code examples often cannot be run as posted online}.
\emph{P3} wrote C++ code using XCode, a development environment for Macintosh platforms.
This environment did not require explicit \emph{std::} scoping for strings that were present in the example he found.
He manually inspected compiler errors to discover why the code would not run, and adapted the code from the online examples for his coding context using this feedback.
\emph{P4} told us of a similar experience in recent history where he found the appropriate function for his task online, but could not determine the right configuration parameters of the function without a coworker's help.

\emph{Users leverage existing code to perform tasks.}
One programmer \emph{P3} found it easier to complete his coding task by opening existing, running code and copying in source code from the answers he found online.
Another \emph{P5} told us that when at work, he was most likely to check the company repository for related code before searching online for solutions.

\emph{Users begin their search from Google.}
Each of the programmers \emph{(P1 - P5)} started in the search bar or in Google to find resources.
Search tools that look to provide an overview over a general collection of code examples from various resources might benefit from comparing themselves to the baseline of Google.

These preliminary studies inspired the design of \systemname{}.
We set the following goals:

\begin{itemize}[noitemsep]
\item Help users distinguish between code-heavy and text-heavy documents.
\item Demonstrate which libraries are core to solving a problem, and provide access to external documentation for these modules.
\item Enable users to quickly locate sections of the code that leverage key libraries.
\item Store code snippets in compact location allowing for later inspection and comparison.
\item Integrate visual features into a UI that supports iterative search.
\end{itemize}