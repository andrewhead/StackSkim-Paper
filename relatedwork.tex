\section{Related Work}

Our work draws upon ideas from 3 lines of inquiry:
helping programmers use the web to solve problems,
writing good technical documentation,
and generating and augmenting tutorials.

\subsection{Helping Programmers Use the Web to Solve Problems}

Recent studies have explored programmers' practice of opportunistic programming~\cite{brandt_two_2009}\cite{brandt_example-centric_2010}\cite{hartmann_hacking_2008}.

Still, both experienced programmers~\cite{duala-ekoko_asking_2012}\cite{duala-ekoko_information_2010} and end-user programmers~\cite{dorn_lost_2013}\cite{dorn_learning_2010}\cite{rosson_everyday_2004} struggle to leverage web documentation to solve programming problems.
\andrew{TODO: read~\cite{duala-ekoko_information_2010}\cite{dorn_learning_2010}\cite{rosson_everyday_2004}.}

Formally, we can look at these programming challenges as Ko et al.'s end-user programming learning barriers~\cite{ko_six_2004}.
For example, users may encounter the $selection$ barriers when trying to find APIs that will help them perform new tasks.
When troubleshooting the program, they may face the $understanding$ barrier to debug system output, or the $information$ barrier to try to learn more about silent failures.
Furthermore, programmers who develop a poor mental model of programming techniques~\cite{winslow_programming_1996} that they find online may find this to be an impediment during later learning.
\andrew{Make sure we're not misquoting the Winslow reference.}

Over the last couple of decades, a variety of systems and tools have been developed to support programmer learning in the wild
(i.e. outside the on-campus or online classroom).
Some of these are general purpose tools to promote problem solving and self-led informal learning.
The Idea Garden~\cite{cao_barriers_2012}\cite{cao_end-user_2013} provokes users to answer solve their own problems by asking 3 types of questions designed to help them think and seek new information productively.
\andrew{be specific.}
ScriptABLE~\cite{dorn_scriptable_2011} wraps online tutorials as `informal learning cases' to improve users' abilities to take lessons away from them. 
\andrew{be specific.}
\andrew{Is this actually for programming tasks or just general tasks?}

Additionally, researchers have recommended strategies for improving supporting debugging and development tasks.
Gross \& Kelleher~\cite{gross_non-programmers_2010} recommend strategies for non-programmers to understand unfamiliar code in the Storytelling Alice programming environment.
\andrew{double check}.
In the same line of inquiry, Gross et al. present Dinah~\cite{gross_dinah_2011} to help non-programmers select code for reuse based on its graphical output instead of the source alone.
Grigoreanu~\cite{grigoreanu_end-user_2012} recommend two strategies for sensemaking during end-user debugging.
Webcrystal~\cite{chang_webcrystal_2012} generates explanations of user-selected elements of HTML pages to help webpage authors better understand how to construct similar pages.
The Whyline~\cite{ko_designing_2004} answers programmers `why' questions about program behavior in terms of runtime events in the Alice programming environment.
\andrew{TODO: read~\cite{ko_designing_2004}}

\andrew{Is there anything commercial that we need to consider?}

\subsection{Writing Good Technical Documentation}

Minimal instruction theory provides guidelines for developing usable technical documentation~\cite{carroll_nurnberg_1990}.
There are three main insights of the minimalist approach:
learners are allowed to start immediatedly on meaningfully realistic tasks,
the amount of reading and other passive activity in training is reduced,
and errors and error recovery are presented in a way to make them less traumatic and more pedagogically productive.
Interstingly, today we find ourselves at a where that much of the typical programming documentation is no longer developed by professionals.
Programmers increasingly use \emph{crowd documentation}~\cite{parnin_measuring_2011} found on the web that may be written by hobbyists without formal training writing usable documentation and who do not have QA resources to cross-check documents' quality.
\andrew{Is this the right reference for the term crowd documentation?}

When it is not possible to produce minimal instruction that has been iteratively tested and tailored to its audience, Farkas recommends \emph{layered documentation}~\cite{farkas_layering_1998}.
Layered documentation allows users to access more \emph{backup information} for tasks like error recognition and correction, and enables the same documentation to be used by readers from different backgrounds.
We position our work with the core belief that minimal instruction is a worthwhile but likely unattainable standard for online programming tutorials.
Given the diverse audiences of programming tutorials and the lack of documentation expertise of their authors, minimal instruction seems impossible to achieve.
We therefore adapt Farkas's advice with a technique for adding interactive layering to existing tutorial documentation on the web.
Through this, we approach the aims of minimal instruction: improved transfer of tutorial skills to personal tasks, less web search for discovering background knowledge, and faster error recovery.

\andrew{Read and determine if any of the following are relevant~\cite{eiriksdottir_procedural_2011}\cite{lau_interpreting_2009}\cite{ames_just_2001}.}

\subsection{Generating and Augmenting Tutorials}

Given users' growing dependence on tutorials for learning how to perform workflows in creative tools, programming, and beyond, there has been recent interest in understanding, generating and automatically augmenting tutorials.

\subsubsection{Measuring tutorial quality}

Lafreniere et al.~\cite{lafreniere_understanding_2013} determined the type of activites that users engage in through comment-based discussion following the body of tutorials.
\andrew{be specific.  What were the activities, how do people learn from tutorials, what are their weaknesses?}
In the programming domain, Parnin \& Treude~\cite{parnin_measuring_2011} find that 87.4\% of the methods of the jQuery API are described by a blog post in the first 10 web search results.
Of these blog posts, about half were tutorials.
Toward a Calculus of Confidence
Characterizing Web-Based Tutorials: Exploring Quality, Community, and Showcasing Strategies

\subsubsection{Detecting and processing tutorial structure}
Then click ok!: extracting references to interface elements in online documentation
Browsing and analyzing the command-level structure of large collections of image manipulation tutorials
Delta: A Tool for Representing and Comparing Workflows
\andrew{I'm positive there's work here that I've missed.}

\subsubsection{Authoring and generating tutorials}
Authoring Multi-Stage Code Examples with Editable Histories
Democut: generating concise instructional videos for physical demonstrations
Automatically generating tutorials to enable middle school children to learn programming independently
MixT: automatic generation of step-by-step mixed media tutorials
\andrew{I'm positive there's work here that I've missed.}

\subsubsection{Improving existing tutorials}
Learnersourcing subgoal labeling to support learning from how-to videos
Community enhanced tutorials: improving tutorials with multiple demonstrations
