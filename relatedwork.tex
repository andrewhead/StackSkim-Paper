\section{Related Work}

Our work is related to research from two themes.
In this paper, we compare \Glspl{name} to past work on the automatic explanation of code and demonstration of code.

\subsection{Automatic Explanation of Code}

Most closely related to our work is Webcrystal~\cite{chang_webcrystal_2012}, a tool that assists web authors with low-level programming tasks required to reuse and learn from examples.  
Webcrystal generates human-readable textual answers to user's ``how'' questions about how to recreate aspects of selected HTML elements as well as customized code snippets for the users to recreate the design.
Similarly to Webcrystal, we generate human-readable representations of code to help programmers reuse and learn from online examples.
In contrast to Webcrystal, we focus on describing short, embedded languages like regular expressions and Unix commands instead of HTML, and develop guidelines for generating effective explanations for these languages.

In its technical approach, our work relates to recent efforts to generate natural language explanations of code and software engineering artifacts~\cite{sridhara_automatically_2011,burden_natural_2011,sridhara_towards_2010,kamimura_towards_2013,mcburney_automatic_2014,sridhara_generating_2011,moreno_automatic_2013}.
This body of work aims to reduce the cost of searching and understanding code and specifications by automatically summarizing blocks of code~\cite{sridhara_automatically_2011}, class diagrams~\cite{burden_natural_2011}, Java methods~\cite{sridhara_towards_2010}, unit test cases~\cite{kamimura_towards_2013}, method context~\cite{mcburney_automatic_2014}, parameters~\cite{sridhara_generating_2011}, and classes~\cite{moreno_automatic_2013}.
Similarly to Sridhara et al., we consider how to detect explainable code prior to generating descriptions~\cite{sridhara_automatically_2011,sridhara_towards_2010}.
However, their code detection process emphasizes summarization of methods and blocks of code rather than locating instances of a language in potentially mixed-language code snippets.
Our work is also distinguished from this past work by examining several frequently used languages from web-based programming help.
%%This enables us to test out and propose guidelines for detecting, parsing and explaining code by considering the syntax, purpose, and usage scenarios of each language.

\subsection{Automatic Demonstration of Code}

%Software visualization has a long history of producing visual representations of code for many purposes~\cite{sorva_visual_2012}.
Visual tools have been used to aid in programming instruction; Sorva \cite{sorva_visual_2012} presents a good review.
PythonTutor~\cite{guo_online_2013} is a recent programming visualization tool for CS education.
The tutor is embeddable within a web page and supports simultaneous viewing of program source code, program execution, visual representations of Python objects, and program output.
More recently, Ou et al.\ produced visualizations of pointer-based data structures in the heap~\cite{ou_interactive_2015}.
Beyond visualizations, code can be represented by discrete structures.
D'Antoni et al.~\cite{dantoni_how_2015}\ created counterexamples of system behaviors to provide helpful feedback for computer science students doing automata design problems.
We draw inspiration broadly from this past work in visualization and example generation to automatically produce demonstrations of regular expressions and CSS selectors.
\shortchange{We believe we are the first to produce guidelines for generating effective \glspl{exp} of code found in online programming help.}
% consider building demonstrations from the perspective of writing design guidelines for micro-explanations of code found in online programming help. \marti{what is meant by "design guidelines" here?}
