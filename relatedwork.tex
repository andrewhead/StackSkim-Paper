\section{Related Work}

Most closely related, Webcrystal~\cite{chang_webcrystal_2012} generates explanations of user-selected elements of HTML pages to help webpage authors learn how to construct similar elements on their own web pages.
Gross \& Kelleher~\cite{gross_non-programmers_2010} recommend strategies for non-programmers to understand unfamiliar code in the Storytelling Alice programming environment.
\andrew{This needs a better explanation.}

A great deal of work has been done in the last five years automatically generating natural language explanations of code and software engineering artifacts~\cite{sridhara_automatically_2011,burden_natural_2011,sridhara_towards_2010,kamimura_towards_2013,mcburney_automatic_2014,sridhara_generating_2011,haiduc_supporting_2010,moreno_automatic_2013}.
Much of this work has focused on how to reduce the amount of time developers spend reading poorly-documented code by automatically summarizing Java methods~\cite{sridhara_towards_2010}, blocks of code~\cite{sridhara_automatically_2011}, method parameters~\cite{sridhara_generating_2011}, and classes~\cite{moreno_automatic_2013}.
Similar to this work, our system takes code as input and provides a natural language description of the procedures followed.

In our own research, we leverage insights from this past work that has focused on improving software maintenance tasks.
Our ultimate goal is to help end user and novice programmers understand short snippets of unfamiliar code online.
We believe we are the first to examine the problem of explaining code in this light.
This problem is challenging in its own right.
First, we examine syntactic patterns of micro-language structure beyond Java and recommend methods to explain these languages.
Second, we introduce a method of planning and implementing explanations of multiple levels of detail based on usage information mined from the web.

Additionally related to the problem of explaining code are the following systems.
PythonTutor visualizes program output for arbitrary Python programs~\cite{guo_online_2013}.
\andrew{An Interactive System for Data Structure Development: visualizing arbitrary pointer-based data structures}
Outside of programming, O'Rourke et al.'s automatically generating interactive instructional demonstrations that interleaves natural language explanations of procedural steps of math problems in code for problem-solving algorithms.~\cite{orourke_framework_2015}.
Similarly, we envision \gls{name} developers opportunistically making use of existing parsing code to build syntax trees or lists of arguments that can be used to generate explanations.
\andrew{Active Code Completion: library-specific visualizations of code structures.}

We also see our work supporting a larger effort to help programmers make use of and learn how to program in unstrutured, out-of-classroom contexts.
\andrew{Elaborate on the lessons learned from each of these projects}.
The Idea Garden~\cite{cao_barriers_2012}\cite{cao_end-user_2013} provokes users to answer solve their own problems by asking 3 types of questions designed to help them think and seek new information productively.
ScriptABLE~\cite{dorn_scriptable_2011} wraps online tutorials as `informal learning cases' to improve users' abilities to take lessons away from them. 
