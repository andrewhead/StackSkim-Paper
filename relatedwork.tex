\section{Related Work}

\subsection{Automatically Generating Code Explanations}

\andrew{TODO: read all in this paragraph.}
A great deal of work has been done in the last five years automatically generating natural language explanations of code~\cite{sridhara_automatically_2011,burden_natural_2011,sridhara_towards_2010,kamimura_towards_2013,mcburney_automatic_2014,sridhara_generating_2011,haiduc_supporting_2010,moreno_automatic_2013}.
Similar to this work, our system takes code as input and provides a natural language description of the procedures followed.

In our own research, we leverage insights from this past work that has focused on improving software maintenance tasks.
Our ultimate goal is to help end user and novice programmers understand short snippets of unfamiliar code online.
We believe we are the first to examine the problem of explaining code in this light.
This problem is challenging in its own right.
First, we examine syntactic patterns of micro-language structure beyond Java and recommend methods to explain these languages.
Second, we introduce a method of planning and implementing explanations of multiple levels of detail based on usage information mined from the web.

PythonTutor visualizes program output for arbitrary Python programs~\cite{guo_online_2013}.
\andrew{An Interactive System for Data Structure Development: visualizing arbitrary pointer-based data structures}
Outside of programming, O'Rourke et al.'s automatically generating interactive instructional demonstrations that interleaves natural language explanations of procedural steps of math problems in code for problem-solving algorithms.~\cite{orourke_framework_2015}.
Similarly, we envision \gls{name} developers opportunistically making use of existing parsing code to build syntax trees or lists of arguments that can be used to generate explanations.
\andrew{Active Code Completion: library-specific visualizations of code structures.}

\subsection{Helping Programmers Use the Web to Solve Problems}

Over the last couple of decades, a variety of systems and tools have been developed to support programmer learning in the wild
(i.e. outside the on-campus or online classroom).
Some of these are general purpose tools to promote problem solving and self-led informal learning.
The Idea Garden~\cite{cao_barriers_2012}\cite{cao_end-user_2013} provokes users to answer solve their own problems by asking 3 types of questions designed to help them think and seek new information productively.
\andrew{be specific.}
ScriptABLE~\cite{dorn_scriptable_2011} wraps online tutorials as `informal learning cases' to improve users' abilities to take lessons away from them. 
\andrew{be specific.}
\andrew{Is this actually for programming tasks or just general tasks?}

Additionally, researchers have recommended strategies for improving supporting debugging and development tasks.
Most closely related, Webcrystal~\cite{chang_webcrystal_2012} generates explanations of user-selected elements of HTML pages to help webpage authors better understand how to construct similar pages.
Gross \& Kelleher~\cite{gross_non-programmers_2010} recommend strategies for non-programmers to understand unfamiliar code in the Storytelling Alice programming environment.
\andrew{double check}.
In the same line of inquiry, Gross et al. present Dinah~\cite{gross_dinah_2011} to help non-programmers select code for reuse based on its graphical output instead of the source alone.
The Whyline~\cite{ko_designing_2004} answers programmers `why did' questions about program behavior in terms of runtime events in the Alice programming environment.
