\section{Related Work}

Programmers rely on application programming interfaces (APIs) to provide the building blocks for their programs from simple math to advanced networking and multi-threading.
The task of remembering many APIs produces a high cognitive demand.
As a result, the web has become an on-demand resource for both learning and frequently recalling API functionality and details.
Programmers encounter similar challenges across the unfamiliar APIs they use~\cite{Duala-Ekoko-Asking}.
% Not all APIs are equally easy to use, however, necessitating new ways of evaluating and visualizing API complexity~\cite{de_Souza-Automatic}.

Recent research describes a current trend in programming behavior called `opportunistic programming'.
Under this method, programmers leverage just-in-time resources found on the web as they build code~\cite{Brandt-Opportunistic}.
Tools have been built~\cite{Hartmann-Dmix} to help programmers leverage web resources when creating `mashup' of existing services.
This trend of interleaved development and information foraging has been studied in specific programmer groups including web developers~\cite{Dorn-Interleaved}.

A web search for API documentation returns results of many types, including Q\&A, forum entries, mailing list entries, official reference documentation, and blog posts such as tutorials~\cite{Parnin-Measuring}.
StackOverflow has recently received attention in a source of ``crowd documentation''~\cite{Parnin-Crowd} for the quality of information and speed of response to questions~\cite{Mamykina-Fastest}, its high coverage of API methods~\cite{Parnin-Crowd}, and variety of question types~\cite{Nasehi-What}.
The success of StackExchange-style sites has attracted open source projects to establish a major presence on the site~\cite{Vasilescu-How}.
However, even with recent advances in quality of crowd documentation, end-user programmers still stumble to find relevant, usable resources and to adapt snippets found to their programs.

Among the models of information seeking on the web, the berry-picking model~\cite{Bates-Design} seems to describe the snippet search process well.
Users continually refine their query and adjust their information need, in this case a functional programming block, as they encounter intermediate search results.
During this process, programmers may face the vocabulary problem~\cite{Furnas-Vocabulary}, not knowing the query terms that represent the code they're looking for.
And even when they find these resources, end-user programmers may still struggle to adapt them for personal use.
Ko et al.~\cite{Ko-Six} introduced six learning barriers for end-user programming tools, several of which are related to search for code examples:
\begin{itemize}
\item $Selection$: knowing which classes and functions are the best ones to choose
\item $Coordination$: joining the functionality of multiple snippets into a coherent whole
\item $Understanding$: developing enough of an understanding of the inner workings of the code and classes chosen to know when they aren't working and why.
\end{itemize}

Specialized code search interfaces Mica~\cite{Stylos-Mica}, Assieme~\cite{Hoffman-Assieme}, and Portfolio~\cite{McMillan-Portfolio} attempt to help users find resources that are both relevant~\cite{Stylos-Mica}~\cite{Hoffman-Assieme} and plentiful~\cite{McMillan-Portfolio}.
Others have integrated the code search and adoption process into the programmer's development environment (IDE).
Blueprint~\cite{Brandt-Example} brings snippet search into the IDE through an embedded browser.
HyperSource~\cite{Hartmann-HyperSource} links recent websites visited to lines of code, and CodeTrail~\cite{Goldman-CodeTrail} provides access to Javadocs and stores user-created bookmarks.
SnipMatch~\cite{Wightman-SnipMatch}, perhaps closest in intent to the current research, adapts found code snippets to the programmer's IDE, replacing template elements with specific symbols from the programmer's code.
However, this tool requires humans to manually creation snippet templates.
It does not help the process of adapting snippets that already exist among the 15 million answers on StackOverflow as of 2015.
