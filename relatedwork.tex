\section{Related Work}

Programmers rely on application programming interfaces (APIs) as the building blocks for their programs.
APIs provide pre-written functionality for a language, enabling tasks from simple math to advanced networking and multi-threading.
As programmers today work with dozens or hundreds of APIs, the web has become an on-demand resource for learning and recalling API functionality and details.
Unfortunately, programmers face challenges as they familiarize themselves with APIs~\cite{Duala-Ekoko-Asking}, and not all APIs are equally usable~\cite{de_Souza-Automatic}.

Recent research describes a common programming behavior called `opportunistic programming'.
Programmers `opportunistically' find web resources to enable the prototyping, ideating and discovering on the web.
Through this process, they code to engage in just-in-time learning, clarify and extend existing knowledge, and remind themselves of forgotten details~\cite{Brandt-Opportunistic}.

End-user programmers write programs or scripts for personal efficiency or effectivness, but lack formal training or employment as a programmer.
Web developers, one group of end-user programmers, have been shown to interleave development and information foraging~\cite{Dorn-Interleaved}.
Tools like d.mix~\cite{Hartmann-Dmix} help such programmers, in particular web programmers, leverage web resources when creating `mashup' of existing services.
However, end-user programmers still face numerous challenges learning from and working with web documentation, being largely unsuccessful in information foraging tasks~\cite{Dorn-Lost}.

A web search for API documentation returns results of many types, including Q\&A, forum entries, mailing list entries, official reference documentation, and blog posts such as tutorials~\cite{Parnin-Measuring}.
StackOverflow has recently received attention in a source of ``crowd documentation''~\cite{Parnin-Crowd} for the quality of information and speed of response to questions~\cite{Mamykina-Fastest}, its high coverage of API methods~\cite{Parnin-Crowd}, and variety of question types~\cite{Nasehi-What}.
The success of StackExchange-style sites has attracted open source projects to establish a major presence on the site~\cite{Vasilescu-How}.
However, even with recent advances in quality of crowd documentation, end-user programmers still stumble to find relevant, usable resources and to adapt snippets found to their programs.

Among the models of information seeking on the web, the berry-picking model~\cite{Bates-Design} seems to describe the snippet search process well.
In this model, users adjust their information need, in this case a need for functional code, as they encounter intermediate search results.
End-user programmers especially may face the vocabulary problem described for human-system communication by Furnas et al.~\cite{Furnas-Vocabulary}, lacking the exact query terms that represent code that will perform the actions they need.
Even after they find appropriate snippets, end-user programmers may still struggle to adapt them to their code.
Ko et al.~\cite{Ko-Six} described six learning barriers for end-user programming tools, at least three of which are relevant to searching for code examples:
\begin{itemize}
\item $Selection$: knowing which classes and functions are appropriate for a task
\item $Coordination$: knowing how to join multiple snippets to perform a larger task
\item $Understanding$: understanding the inner workings of the code and classes chosen enough to know when they aren't working and why.
\end{itemize}

Specialized code search interfaces Mica~\cite{Stylos-Mica}, Assieme~\cite{Hoffman-Assieme}, and Portfolio~\cite{McMillan-Portfolio} help users find resources that are both relevant~\cite{Stylos-Mica}~\cite{Hoffman-Assieme} and many in number~\cite{McMillan-Portfolio}.
Others integrated code search into the programmer's development environment (IDE).
Blueprint~\cite{Brandt-Example} brings snippet search into the IDE through an embedded browser.
HyperSource~\cite{Hartmann-HyperSource} links recent websites visited to lines of code, and CodeTrail~\cite{Goldman-CodeTrail} provides access to Javadocs and stores user-created bookmarks.
SnipMatch~\cite{Wightman-SnipMatch}, perhaps closest in intent to the current research, adapts searchable code snippets to the programmer's code, replacing template elements with specific symbols from the programmer's code.
However, SnipMatch requires humans to manually creation snippet content.
We aim to help users adapt and work with the 15 million answers already written on StackOverflow as of early 2015.
