\section{Related Work}

\subsection{Visual Search}

There have been several attempts to represent features in large text corpora to aid search tasks.
TileBars~\cite{hearst_tilebars:_1995} uses compact iconic displays to represent relative document length, query term frequency, and query term distribution, enhancing a user's ability to quickly judge relevance of a text document. 
Relation Browser~\cite{zhang_evaluation_2005} facilitates faceted visualization of large information collections. 
Insyder~\cite{reiterer_insyder_2000} combines several previously-studied visualizations in one interface to support information seeking of general web content.
StackSkim similarly represents features of a large text corpus, focusing on the domain of programming examples.

In the domain of code visualization, SeeSoft~\cite{eick_graphically_1994-1} uses a similar colored column encoding of multiple source code texts rendering a compact display to reveal statistics about text. 
Tools for supporting web code search like Mica~\cite{stylos_mica:_2006} combine documentation and source code but continue to represent results primarily with text. 
Similarly, Assieme~\cite{hoffmann_assieme:_2007} presents API information from several web resources by using a novel technique for finding implicit references in code. 
Blueprint~\cite{brandt_example-centric_2010} helps users find and integrate code directly in the programmers coding environment.

\subsection{Search for Programming Learners}

Past work has observed programmer search tasks and interfaces to aid these tasks. 
Visualizations and interfaces have been produced for analyzing learning algorithms~\cite{hundhausen_meta-study_2002}, program execution~\cite{de_pauw_zinsight:_2010}~\cite{aftandilian_heapviz:_2010}, program comprehension~\cite{islam_dependence_2010} and program maintenance~\cite{bragdon_code_2010}~\cite{coblenz_jasper:_2006}. 
Search behavior and interfaces have also been studied in the domains of craft knowledge~\cite{torrey_learning_2009} and selecting tutorials~\cite{Pavel:EECS-2013-167}~\cite{kong_delta:_2012}.

Ko et al.\ categorize six types of programming learning barriers - design, selection, coordination, use, understanding and information~\cite{ko_six_2004}. 
Our \systemname{} interface primarily addresses \emph{selection} and \emph{use} barriers in the terminology of Ko et al. 
Our answer bars and concept and class charts allow programmers to quickly identify most commonly used classes and concepts and, as a result, what to use to solve a programming problem (\emph{selection barrier}).
Furthermore, the ability to annotate specific answer bars with classes and concepts helps searchers view a variety of ways in which a programming construct is used (\emph{use barrier}).

Brandt et al.\ extend the work of Ko et al.\ by describing how programmers overcome these barriers~\cite{brandt_two_2009}~\cite{ko_six_2004}. 
They identify three types of intentions when foraging for information on the Web: just-in-time learning of new skills, clarifying and extending existing knowledge, and assisting with recall of specific information.
They observe that programmers typically open several Web browser tabs before evaluating quality, rapidly skimming through their content, sometimes skipping prose to begin experimenting with the code sections. 
\systemname{}'s color-coded answer bars provide a quick way for users to view and zone in on the code sections of many answers.

Research on the StackOverflow community has revealed that users benefit significantly from multiple answers from a diverse group of experts~\cite{anderson_discovering_2012}. 
Furthermore, the vertical position of answers positively biases early answers toward best answer status ~\cite{anderson_discovering_2012}.
One of the reasons cited for the success of StackOverflow is its ability to capture the competitive spirit of the programming community through gamification~\cite{mamykina_design_2011-1}.
However these built in incentives can make distinguishing experts from very active users difficult~\cite{yang_sparrows_2014}. 
By making the content of many StackOverflow answers accessible, \systemname{} alleviates some of these issues. 
In addition, we believe \systemname{} better facilitates the exploration of answers to some types of questions that require significant contextual knowledge or obscure technologies which have previously been identified as being poorly supported by StackOverflow~\cite{mamykina_design_2011-1}.